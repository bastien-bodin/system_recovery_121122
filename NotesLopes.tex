\documentclass[10pt,a4paper, twocolumns]{article}
\usepackage[utf8]{inputenc}
\usepackage[french]{babel}
\usepackage[T1]{fontenc}
\usepackage{amsmath}
\usepackage{amsfonts}
\usepackage{amssymb}
\usepackage{graphicx}
\usepackage{lmodern}
\usepackage{gensymb} %symbole \degree
\usepackage[left=1.5cm,right=1.5cm,top=2cm,bottom=2cm]{geometry}
\author{Bastien BODIN\\ Adapté de RMC Lopes, 1991}
\title{Extraterrestrial lava flows}


\begin{document}

\twocolumn

\maketitle
\section{Objectif du papier}
\begin{itemize}
	\item Passer en revue les connaissances actuelles des écoulements de lave sur la Lune, Mars, Vénus et Io, à partir :
	\begin{itemize}
		\item des données disponibles ;
		\item des modèles appliqués sur chacun d'eux.
	\end{itemize}
	\item Discuter des prochaines missions.
\end{itemize}

\section{Introduction}

Différentes caractéristiques d'activité volcanique (passée ou actuelle) ont été identifiées par les satellites d'observation (écoulements de lave, volcans de type basaltique, volcanisme de glace, ...)

Dans ce papier, il sera question de :
\begin{itemize}
	\item volcanisme actif, comme sur Io, où celui-ci pourrait être à base de sulfure, ou encore sur Triton, où des geysers auraient été identifiés ;
	\item volcanisme passé, comme sur la Lune ou encore Mars
	\item volcanisme indéterminé, comme pour Vénus.
\end{itemize}

Il sera également montré que la tectonique des plaques des différents corps étudiés ont un rôle important  dans la morphologie des volcans retrouvés (ex : plus gros sur Mars)

Toutes les données présentées dans ce papier sont des données morphologiques obtenues par imagerie. Quelques données ont cependant été obtenues par analyse spectrales, mais celles-ci sont limitées.

\section{La Lune}

\begin{figure}[!h]
	\centering
	\includegraphics[height=4cm]{images/lune/lune_mare.jpg} 
	\caption{Face cachée de la Lune, les zones sombres sont les maria}
\end{figure}

La première caractéristique identifiée est appelé mare\footnote{Zones sombres sur la surface de la Lune}. Les maria sont le résultat d'éruptions basaltiques similaires à des inondations. Elles sont situées le plus souvent sur la face cachée de la Lune. Quelques fronts d'écoulement ont été aperçus, mais l'identification de fronts individuels est extrêmement ardue.

\begin{figure}[!h]
	\centering
	\includegraphics[scale=0.1]{images/lune/luna_vent.png}
	\caption{Probablement une cheminée volcanique dans la partie sud-ouest du bassin oriental. La largeur de l'image est d'environ 750 m. Image LROC NAC M1150135366 [NASA/GSFC/Arizona State University].}
\end{figure}

Une seconde caractéristique est la cheminée volcanique. L'hypothèse actuelle est que ce sont d'anciennes fissures recouvertes par la lave. Les fronts d'écoulement n'ont pas été conservés, soit parce que la lave était trop fluide, soit parce que les impacts météoritiques étaient trop fréquents. Il est donc difficile d'identifier leurs sources.

L'épaisseur estimée pour les écoulements de mare sont de $\sim$10 de mètres (confirmé par les expériences).

Différentes modélisations de la rhéologie des écoulements lunaires ont été réalisées\footnote{Hulme (1973, 1974) \& Hulme et Fielder (1977)}\footnote{Moore \& Schaber (1975)} :

\begin{table}[h]
%Hulme (1973, 1974) \& Hulme et Fielder (1977)
%Moore \& Schaber (1975)
	\centering
	\begin{tabular}{|c||c|c|c|c|}
		\hline
		Réf.      & Modèle  &                & Mare              & Écoul.\\
		          & utilisé &                & Imbrium           & d'impacts  \\
		\hline
		Hulme     & Bingham & $\tau_0\ [Pa]$ & $400$             & $2.10^4$\\
		          &         & $Q\ [m^3/s]$   & $2.10^4 - 8.10^4$ & $260 - 1000$\\
		\hline
		Moore     & Bingham & $\tau_0\ [Pa]$ & $100-200$         & $2.10^4$\\
		Schaber   &         & $Q\ [m^3/s]$   & X                 & X \\
		\hline
	\end{tabular}
\end{table}
Limites :
\begin{itemize}
	\item Aucune preuve que le modèle de Bingham soit valide pour les écoulements de lave ;
	\item Les écoulement lunaires sont supposés similaires aux écoulements basaltiques terrestres
\end{itemize}

\subsection{Analyse des échantillons}
	\begin{itemize}
		\item Viscosité des maria: $\sim 1 Pa.s$ à 1400 \degree C + faibles pertes calorifiques $\Rightarrow$ écoulement sur de longues distances
		\item Composition : basalte $\neq$ basaltes terrestres. $\emptyset\ H_2O$, grande concentration de $Fe, Mg, Ti \Rightarrow$ flaible concentration de silicate.
		\item Région Oceanus Procellarum : Élements $K$, $REE$\footnote{REE : Rare Earth Elements} et $P$
		\item Pas d'altérations dues au climat.
		\item La plupart des laves viennent d'environ 150-450 km sous la surface, il y a $\sim 3,1.10^9-3,9.10^9$ ans.
	\end{itemize}
	
\subsection{Caractéristiques de petite échelle}

	\begin{itemize}
		\item Rainures sinueuses :
		\begin{itemize}
			\item des canaux sinueux qui peuvent avoir une excavation sans rebord à une extrémité ;
			\item principalement sur les extrémités des maria ;
			\item peut être utilisé pour identifier les directions de l'écoulement ;
			\item taille : environ 130 km de long, 5 km de large.
		\end{itemize}
		\item Crêtes de mare :
		\begin{itemize}
			\item $\sim 10$ km de long ;
			\item caractéristiques compressives ;
			\item corrélé avec le lieu de la plupart des séquences de basalte de mare
		\end{itemize}
	\end{itemize}
	
	\begin{figure}[!h]
		\centering
		\includegraphics[height=4cm]{images/lune/sinuous_rilles.jpg} 
		\includegraphics[height=4cm]{images/lune/wrinke_ridges.jpg} 
		\caption{Une partie de l'image LROC M115429448L (résolution de 0,970 m/pixel) montrant un gros plan d'une rille sinueuse (flèches) qui traverse des plaines sombres (p) et des matériaux collinaires adjacents (h) sur le plancher de Schrödinger. (GAUCHE)\\
		Longues crêtes dans Mare Tranquillitatis sur l'image M117345275M de la caméra grand angle (WAC) du LRO. La section de la crête ridée montrée dans l'image du haut est au centre de cette vue WAC. Le cratère Arago (en bas à gauche), a un diamètre de 26 km. Crédit : NASA/Goddard/Arizona State University.(DROITE)}
	\end{figure}

\subsection{Perspectives d'acquisition de données}

	Les meilleures perspectives seront obtenues principalement par une série de travaux effectués au sol :
	\begin{itemize}
		\item études spectrales (en laboratoire, télescopiques) pour étudier la minéralogie lunaire ;
		\item Lunar Observer, qui permettra de mener des études sur la géochimie et la géophysique ;
		\item l'amélioration des modèles d'éruptions de basalte.
	\end{itemize}

\section{Mars}

\subsection{Unités géologiques}

	Différence d'élévation Nord/Sud : le Sud est 1-3 km au dessus du datum\footnote{Datum : altitude de référence}
	
	\begin{figure}[!h]
		\centering
		\includegraphics[height=3.9cm]{images/mars/Mars_topography_(MOLA_dataset)_with_poles_HiRes.jpg}
		\caption{Carte topographique haute résolution de Mars basée sur les recherches menées par Maria Zuber et David Smith sur l'altimètre laser de Mars Global Surveyor. Le nord est en haut. Les caractéristiques notables comprennent les volcans de Tharsis à l'ouest (y compris Olympus Mons), Valles Marineris à l'est de Tharsis, et le bassin Hellas dans l'hémisphère sud.}
	\end{figure}
	
	Identifié par :
	\begin{itemize}
		\item Dans l'hémisphère nord, des plaines de lave relativement jeunes, et des caractéristiques volcaniques ;
		\item Dans l'hémisphère sud, un terrain plutôt ancien, avec de nombreux cratères, des caractéristiques volcaniques érodées.
	\end{itemize}
	
	Dans les régions volcaniques de l'hémisphère nord, on retrouve les plus gros édifices, tels que :
	\begin{itemize}
		\item La région Tharsis :
		\begin{itemize}
			\item 8 000 km de diamètre ;
			\item 10 km d'altitude au dessus du datum.
		\end{itemize}
		\item Elysium, un plus petit renflement ;
		\item Olympus Mons (dans la région Tharsis) :
		\begin{itemize}
			\item 25 km d'altitude ;
			\item 600 km  de diamètre ;
			\item flancs peu prononcés (4 - 6\degree ) ;
			\item caldeiras\footnote{Caldeira : Grande dépression formée par l'effondrement de la partie supérieure du cône d'un volcan à la suite d'éruptions intenses et rapides.} complexes ;
			\item de nombreux écoulements de lave.
		\end{itemize}
	\end{itemize}
	Ces éléments montrent notamment l'importance de l'évolution tectonique de Mars.
	
	Quelques éléments identifiés :
	\begin{itemize}
		\item Tholii\footnote{Tholii : domes} : petits volcans originant peut-être d'une activité explosive ;
		\item Plaines volcaniques : rainures sinueuses, crêtes, lobes d'écoulements se recouvrant ;
		\item Alba Patera\footnote{Patera : Particularité en forme de soucoupe} : plus vaste volcan bouclier de Mars :
		\begin{itemize}
			\item 1 600 km de diamètre ;
			\item 6 km d'altitude ;
			\item La plupart des écoulements sont visibles :
			\begin{itemize}
				\item écoulements en nappe ($\sim$ dizaine de km);
				\item écoulements à crête ;
				\item écoulements tubulaires, de larges levées sont observées ;
			\end{itemize}
		\end{itemize}
	\end{itemize}

\subsection{Étude des écoulements de lave}

	\begin{table}[!h]
	%refs a mettre
		\centering
		\begin{tabular}{|c||c|c|c|}
			\hline
			Réf.       &   Param.       & Olympus             & Conclusion  \\
			           &                & Mons                &             \\
			\hline
			Hulme      & $\tau_0\ [Pa]$ & $3,9.10^3-2,3.10^4$ & Lave de     \\
			           & $Q\ [m^3/s]$   & $380 - 470      $   & silice      \\
			\hline
			\hline
			Réf.       &  Param.        & Écoulement          & Conclusion  \\
			           &                & de levée            &             \\
			\hline
			Carr       & $\tau_0\ [Pa]$ & $10^2-10^3$         & Seulement   \\
			et al      & $Q\ [m^3/s]$   & $10^3-10^4$         & silice ?    \\
			\hline
		\end{tabular}
	\end{table}

	Wadge \& Lopes (1991), étude des lobes d'écoulements de lave :
	\begin{itemize}
		\item corrélation largeur des lobes/teneur en silice ;
		\item écoulements sur Olympus Mons $\sim$ écoulements terrestres avec une teneur en andésite ou une mixture basalte/silice.
	\end{itemize}

	Crisp \& Baloga (1990) ont calculé un taux d'effusion $Q\sim2.10^4\ m^3/s$
	
	Une autre méthode, plus intéressante pour les laves extraterrestres car indépendante de la gravité ou de la viscosité, a été utilisée par Kilburn \& Lopes (1991) sur les écoulements de Alba Patera, et ont trouvé $Q \sim 10^4-10^5\ m^3/s$.
	
	Seulement, il a été trouvé $Q/L_{fissure} \sim 5-15\ m^2/s$, ce qui correspond aux éruptions basaltiques sur Terre. Cela montre que le lien entre taux d'effusion et longueur des fissures ne peut pas être utilisé pour en déduire la composition des laves
	
\subsection{Prochaines acquisitions}

	\begin{itemize}
		\item Mesures fiables des écoulements : Mars Observer (résolution : 1,4m/pixel ;\\MOLA : résolution de 1,5m);
		\item Composition chimique : ramener des échantillons sur Terre ;
		\item Réalisation d'expériences sur place.
	\end{itemize}

\section{Vénus}

	\subsection{Types de structures volcaniques}
	
	$\sim$ 80\% de la surface peut être associée à des structures volcaniques, le reste pouvant être assimilé à des régions déformées par l'activité tectonique.
	
	\begin{figure}[!h]
		\centering
		\includegraphics[height=4cm]{images/venus/pioneer_venus_map_of_venus.jpg} 
		\caption{Une carte de Vénus obtenue à partir des données enregistrées par le satellite Pioneer Venus Orbiter de la NASA en 1978.}
	\end{figure}
	
	\begin{itemize}
		\item Structure la plus commune : petits dômes (200m d'altitude, 2-8 km de diamètre)
		\item Sif Mons :
		\begin{itemize}
			\item 300 km de diamètre, 1,7 km d'altitude
			\item un écoulement de 300 km de long, 15-30 km de large
		\end{itemize}
	\end{itemize}
	
	Composition :
	\begin{itemize}
		\item Missions soviétiques :
		\begin{itemize}
			\item basaltes alcalins/tholéiitiques ?
			\item composition à base de silice ?
		\end{itemize}
		\item Head et al. (1991) concernant les dômes de  Alpha Regio :
		\begin{itemize}
			\item similaires aux dômes andésitiques, rhyolitique ou dacitique\footnote{Dacite : roche composée de quartz} sur Terre, mais plus large, potentiellement à cause d'une température de surface plus élevée ;
		\end{itemize}
	\end{itemize}
	
	\begin{figure}[!h]
		\centering
		\includegraphics[height=4cm]{images/venus/Fotla_Corona_PIA00202.jpg} 
		\caption{Fotla Corona, une corona de 150 km de diamètre située au sud de Dsonkwa Regio, par 58,5° S et 163,5° E [NASA/JPL/Magellan].}
	\end{figure}
	Autres structures :
	\begin{itemize}
		\item Coronæ \footnote{Corona : région en forme d'anneau} : peut-être une remontée de panaches mantelliques provoquant un renflement localisé (environ 200-1 000 km de diamètre).
		\item Écoulements sinueux larges, dûe à une faible viscosité de la lave :
		\begin{itemize}
			\item Similaire à ceux sur la Lune ;
			\item 0,5-1,5 km de large, longueur supérieure à 1 000 km
		\end{itemize}
	\end{itemize}
	
\subsection{Études des écoulements de lave}

	Quelques écoulements sont similaires aux écoulements basaltiques et aux komatiites\footnote{Komatiites : Roches volcaniques faibles en silice (<45\% en masse) et riche en magnésium (20-30\%)}.
	
	\begin{itemize}
		\item Roberts et al. (1992), sur l'étude de Mylitta Fluctus : composition basaltique ;
		\item Beaucoup d'incertitudes : on ne sait pas si les structures sont des $'a\bar{a}$\footnote{$'a\bar{a}$ : type de coulée qui se solidifie rapidement et prend un aspect croûté, acéré et coupant} ou des pahoehoe\footnote{pahoehoe : type de lave, généralement basaltique, parfois carbonatique, partiellement dégazée, très pauvre en silice, ayant l'aspect d'un amas de cordes plissées} ;
		\item Moore et al. (1992) :
		\begin{itemize}
			\item $\tau_0 \sim 2.10^4-4.10^5\ Pa \Rightarrow$ lave basaltique, andésitique ou rhyolitique, mais avec de grandes incertitudes ;
			\item $\mu \sim 1.10^7 - 8.10^9\ Pa.s \Rightarrow$ composition plus évoluée que les roches basaltiques ;
			\item $\Rightarrow$ Le volcanisme effusif basé sur une composition riche en silice était importante.
		\end{itemize}
	\end{itemize}

\section{Io}

Un volcanisme actif est présent à la surface de Io, causé par les forces de marées que ce satellite subit de par sa proximité avec Jupiter.

\begin{figure}[!h]
	\centering
	\includegraphics[height=4cm]{images/io/Tidal_heating_on_Io-fr.png}
	\caption{
Schéma du réchauffement par effet de marée sur Io : (A) La gravité des autres lunes galiléennes influence l'orbite de Io et maintient son excentricité ; (B) Du fait de cette orbite excentrique, la forme de Io change au cours de son orbite.}
\end{figure}

\subsection{Sulfure ou Silicate}

	\subsubsection{1ère hypothèse : Sulfure}
		\begin{itemize}
			\item Observations spectrales et expériences de laboratoire : fine couche de sulfure à la surface ;
			\item Du dioxyde de sulfure a été détecté sur le volcan Loki ;
			\item Du sulfure ionisé a été détecté dans le tore auquel Io appartient ;
			\item Des plumes éruptives seraient des geysers de sulfure avec du $SO_2$.
		\end{itemize}
	
	\subsubsection{Hypothèse privilégiée : Silicate}
	
	\begin{itemize}
		\item Les interprétations basées sur la couleur de surface sont remises en cause, notamment par des calibrations plus précises (Young, 1984) ;
		\item Pour les éruptions de longue durée, les températures peuvent dépasser les 900K (Goguen, 1992, points chauds à $\sim$  1 150K), ce qui est supérieur au point d'ébullition du sulfure, mais cohérent avec une lave silicatée (il est également possible que la lave soit composée de polysulfure de sodium) ;
		\item Ont été observées de hautes montagnes, des falaises abruptes, des murs de caldera (Clow \& Carr, 1980).
		\begin{itemize}
			\item Ce n'est pas cohérent avec une composition de sulfure uniquement, car un comportement plus malléable aurait été observé dans ce cas
			\item En revanche, avec un ajout de silicate dans la composition, cela devient cohérent avec les contraintes mécaniques observées
			\item Une composition dominante de sulfure est également à exclure de par la présence de caldera.
		\end{itemize}
	\end{itemize}
	
	$\Rightarrow$ Un volcanisme basé sur le silicate (avec des traces de sulfure) est donc privilégié.

\subsection{Particularités de surface}
	À ce jour ont été identifiés :
	\begin{itemize}
		\item $\sim$ 300 cheminées ;
		\item de nombreuses calderas ;
		\item des écoulements de lave
		\begin{itemize}
			\item de cratères en cheminée ;
			\item de cratères en bouclier, principalement au niveau de l'équateur ;
			\item de fissure, plus épais que les deux précédents.
		\end{itemize}
	\end{itemize}
	
	Cependant, il y a trop d'incertitudes pour appliquer des modèles à ces relevés.

\subsection{Prochaines études}
	
	\begin{itemize}
		\item Galileo
		\begin{itemize}
			\item études des changements dans la morphologie des calderas, montagnes, écoulement ;
			\item détection de nouvelles particularités ;
			\item New-Infrared Mapping Spectrometer (résolution : 50m) $\rightarrow$ sulfure vs silicate ;
			\item Autres instruments $\rightarrow$ comportement volcanique ;
		\end{itemize}
		\item Autres études sur le sulfure en laboratoire.
	\end{itemize}

\section{Conclusion}

\begin{itemize}
	\item Les études présentées ici sont principalement basées sur les interprétations des laves cartographiées par les satellites d'observation, car ces images représentent la plupart des données disponibles ;
	\item Il est important de se souvenir que les modèles utilisés possèdent de grandes incertitudes, notamment car il est nécessaire de combiner des paramètres physiques, chimiques, géologiques ;
	\item Ces modèles morphologiques nous permettraient à terme de mieux comprendre le volcanisme effusif prenant place sur une planète à l'échelle locale ;
	\item Des améliorations dans nos modèles en utilisant les données terrestres nous permettraient de mieux comprendre les laves extraterrestres.
\end{itemize}


\end{document}